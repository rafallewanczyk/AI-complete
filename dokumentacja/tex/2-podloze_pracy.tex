\newpage % Rozdziały zaczynamy od nowej strony.
\section{Podłoże pracy}
W ciągu ostatnich kilku lat przetwarzanie języka naturalnego bardzo się rozwinęło. Wraz z kolejnymi 
badaniami udało się uzyskać coraz lepsze rezultaty. Jednak dział badający zachowanie tych modeli 
na językach programowania jest nowy, co możemy zaobserwować po zbiorze prac \cite{ml4code}
z nim związanych. Jest w nim dużo miejsca na nowe podejścia oraz badania. W tym rozdziale omówię 
opiszę prace istotne lub podobne do mojej.  


\subsection{Sieci neuronowe w przewidywaniu języków programowania}
Subhasis Das, Chinmayee Shah \cite{contextual_code_completion} porównują ze sobą modele
\begin{itemize}
    \item model z wagami o stałej długości okna (fixed window weight model)
    \item model macierzy wektorów (matrix vector model)
    \item sieć neuronowa z wyprzedzeniem (feed-forward neural network)
    \item model z wyprzedzeniem oraz miękka uwagą (feed-forward model with soft attention)
    \item model rekurencyjny z warstwą GRU
\end{itemize}
Kod wejściowy jest poddany tokenizacji przy pomocy wyrażeń regularnych oraz oceniane na podstawie dokładnego 
dopasowania pierwszego przewidzianego tokenu oraz na podstawie 3 najlepszych sugestii. Do testów 
używane są kody bibliotek Django, Twisted oraz jądra systemu Linux. Połowa plików źródłowych jednego z 
projektów używana jest jako zbiór treningowy natomiast druga połowa jako zbiór walidacyjny. Wszystkie modele
osiągają dokładność przewidzeń równą około 80\% dla 3 najlepszych sugestii, najlepiej radzi sobie model miękka uwagą
z dokładnością 83.6\%. Problem słów poza słownikiem rozwiązany jest przy pomocy słownika przypisującego 
token o nieznanej wartości do słowa które wpisał użytkownik. Moja praca różni się tym, że skupiam się 
wyłącznie na sieciach rekurencyjnych, jako że w powyższej warstwa LSTM nie została uwzględniona. Próbuję 
również uogólnić przewidywany kod poprzez trening na projektach zawierających różne biblioteki aby sprawić
by wtyczka była użyteczna w praktyce. \\

Hellendoorn i Devanbu przeprowadzili eksperyment polegający na wykonaniu 15000 predykcji dla 
środowiska Visual Studio.Jako zbioru danych używają 14000 plików źródłowych w języku Java. 
W swojej pracy porównują skuteczność modelu n-gram z modelami rekurencyjnymi. 
Prezentują również dynamicznie aktualizowane modele n-gram działające w zagnieżdżonym zasięgu, rozwiązując 
w ten sposób problem skończonego słownika oraz znacznie usprawniając sugestie. Rezultaty tej pracy 
pokazują, że pomimo znacznie lepszych wyników sieci rekurencyjnych w zadaniu modelowania języka 
naturalnego, modele n-gram w niektórych przypadkach radzą sobie lepiej z przewidywaniem kodu. Jednym
z przytoczonych przykładów są metody wbudowane w język, dla których głębokie sieci działają lepiej, 
jednak przegrywają przy często występujących, zróżnicowanych,  mało popularnych bibliotekach zewnętrznych, 
w których model n-gram naturalnie radzi sobie lepiej, jednak przegrywa pod innymi względami. Swoje 
eksperymenty przeprowadzają dla stałych wartości hiperparametrów, w swojej pracy chcę również
zająć się strojeniem wybranych modeli. \\

Pythia \cite{pythia} stworzona przez zespół programistów Microsoftu, w którego skład wchodzą Alexey Svyatkovskiy, Shengyu Fu, Ying Zhao, Neel Sundaresan
jest rozszerzeniem do wtyczki IntelliSense w środowisku Visual Studio Code. W swojej pracy testują zastosowanie najwydajniejszych 
na ten moment modeli sieci rekurencyjnych. Jako zbiór treningowy użyte jest 2700 projektów z serwisu github \cite{github}
które wcześniej zostały poddane rozbiorowi na drzewa składniowe. Model osiąga najlepszą dotychczasową skuteczność wynoszącą
92\% dla najlepszych 5 sugestii oraz bardzo dobry czas odpowiedzi w okolicach 100 ms. Eksperymentu tego nie będę w stanie 
dokładnie odtworzyć ze względu na inny przygotowanie danych treningowych oraz przez inny zbiór danych (zbiór wykorzystany przez
Pythie nie został udostępniony). Mimo to planuję zaimplementować oraz ocenić wyróżnioną architekturę w moim eksperymencie. 

\subsection {Modele statystyczne w przewidywaniu języków programowania}
Myroslava Romaniuk \cite{pharo} pokazuje, że same modele statystyczne w tym przypadku n-gram, dokładnie 
unigram oraz bigram radzą sobie 
z automatycznym uzupełnianiem kodu. W swojej pracy usprawnia działanie wtyczki do środowiska programistycznego 
Pharo. Zaproponowany model trenuje na 50 projektach w tym języku osiągając dokładność około 40\%. 
W swojej pracy łącze modele rekurencyjne właśnie z połączonymi modelami unigram oraz bigram.

\subsection {Rozwiązania komercyjne}
W dużej mierze do powstania tej pracy przyczyniły się istniejące już rozwiązania komercyjne. Niestety 
ze względów licencyjnych nie są ujawnione dokładnie mechanizmy stojące za ich działaniem, metody użyte 
do treningu oraz dokładna skuteczność, przez co niemożliwe jest porównanie uzyskanych przeze mnie wyników 
z tymi podejściami. \\

Tabnine \cite{tabnine} jest wtyczką do najpopularniejszych środowisk programistycznych realizującą predykcję kolejnego tokenu w 
większości stosowanych języków programowania. Do jej treningu zostało wykorzystane 2 miliony projektów ze strony github \cite{github}. 
Wtyczka opiera się na GPT-2, które używa architektury transformerów. W jej skład wchodzą również zaimplementowane przez twórców
reguły dotyczące języka. Podejście to jest bardzo nowatorskie przez wykorzystanie jeszcze nie zbadanych dokładnie modeli oraz 
różni się od przedstawionego w tej pracy.\\

Open AI realizuje generowanie kodu na podstawie opisu jego działania w komentarzu. Jest to połączenie zadania zrozumienia 
języka naturalnego przez maszynę z zadaniem klasyfikacji. Słownik zamiast składać sie z pojedyńczych tokenów skłąda sie z całych funkcji 
a na wejściu modelu otrzymujemy sekwencje słów zamiast poprzedzający kod. 

\subsection {Rodzaje sieci}
test test test 

test test 

