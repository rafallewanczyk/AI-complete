%%%%%%%%%%%%%%%%%%%%%%%%%%%%%%%%%%%%%%%%%%%%%%%%%%%%%%%
%% Bachelor's & Master's Thesis Template             %%
%% Copyleft by Artur M. Brodzki & Piotr Woźniak      %%
%% Faculty of Electronics and Information Technology %%
%% Warsaw University of Technology, 2019-2020        %%
%%%%%%%%%%%%%%%%%%%%%%%%%%%%%%%%%%%%%%%%%%%%%%%%%%%%%%%

\documentclass[
    left=2.5cm,         % Sadly, generic margin parameter
    right=2.5cm,        % doesnt't work, as it is
    top=2.5cm,          % superseded by more specific
    bottom=3cm,         % left...bottom parameters.
    bindingoffset=6mm,  % Optional binding offset.
    nohyphenation=false % You may turn off hyphenation, if don't like.
]{eiti/eiti-thesis}

\langpol % Dla języka angielskiego mamy \langeng
\graphicspath{{img/}}             % Katalog z obrazkami.
\addbibresource{bibliografia.bib} % Plik .bib z bibliografią

\begin{document}

%--------------------------------------
% Strona tytułowa
%--------------------------------------
\EngineerThesis % Dla pracy inżynierskiej mamy \EngineerThesis
\instytut{Informatyki}
\kierunek{Informatyka}
\specjalnosc{Inżynieria Systemów Informacyjnych}
\title{
    Generowanie automatycznych podpowiedzi \\w środowiskach do programowania
}
\engtitle{ % Tytuł po angielsku do angielskiego streszczenia
    Generating code autocompletions for Integrated Development Environments 
}
\author{Rafał Lewanczyk}
\album{293140}
\promotor{dr inż. Paweł Zawistowski}
\date{\the\year}
\maketitle

%--------------------------------------
% Streszczenie po polsku
%--------------------------------------
\cleardoublepage % Zaczynamy od nieparzystej strony
\streszczenie \\
Uzupełnianie kodu jest funkcją odpowiedzialną za przewidywanie tego co chce, lub może napisać programista. Poprawnie 
działająca może w znaczący sposób wspomóc oraz uefektywnić jego pracę. Istnieje wiele podejść do tego 
problemu np. przy pomocy metod słownikowych lub parsowania na bieżąco pisanego kodu. 

W tej pracy omawiam podejście polegające na zastosowaniu metod uczenia
maszynowego. Generuje ono posortowaną według prawdopodobieństwa wystąpienia listę sugestii, które mogą zostać wykorzystane 
przez programistę w trakcie pracy nad programem. 

Opisuję badane przeze mnie architektury, oraz porównuję je pod względem skuteczności. Wymieniam 
również napotkane wyzwania. 

Końcowy model został wytrenowany na 9000 plikach w języku Python, pochodzących z publicznie dostępnych 
repozytoriów na serwisie GitHub, oraz oceniony na 4500 plikach. Jego skuteczność wyniosła 
\begin{math}68\%\end{math} dla 5 najlepszych predykcji, a średni czas odpowiedzi 18 ms. 

Model ten został zaimplementowany jako wtyczka do środowiska SublimeText 3. 
\slowakluczowe Uczenie maszynowe, autouzupełnianie, kod źródłowy

%--------------------------------------
% Streszczenie po angielsku
%--------------------------------------
\newpage
\abstract
Code autocompletion is a feature that offers suggestions for what a software developer may want 
to write. Correct suggestions improve efficiency of programmers job. There are a lot of forms 
of code autocompletion, for example using dictionary methods or parsing code online. 

In this paper, I propose approach for code autocompletion using machine learning. It generates 
ranked by its probabilities of occurrences list of suggestions, which can be used by software developer at edit time. 

I describe explored architectures and compare them to each other in order to select the best one. I 
also discuss challenges regarding this approach. 

The final model was trained using 9000 Python source files from publicly available repositories from service
GitHub and evaluated on 4500 files. The evaluation results in \begin{math}68\%\end{math}
accuracy for top 5 suggestions, and mean time of response resulted in 18 ms. 

The system is available as a plugin for integrated development environment SublimeText 3. 
\keywords Machine Learning, Autocomplete, Source Code 

%--------------------------------------
% Oświadczenie o autorstwie
%--------------------------------------
\cleardoublepage  % Zaczynamy od nieparzystej strony
\pagestyle{plain}
\makeauthorship

%--------------------------------------
% Spis treści
%--------------------------------------
\cleardoublepage % Zaczynamy od nieparzystej strony
\tableofcontents

%--------------------------------------
% Rozdziały
%--------------------------------------
\cleardoublepage % Zaczynamy od nieparzystej strony
\pagestyle{headings}

\newpage % Rozdziały zaczynamy od nowej strony.
\section{Wstęp}
\label{sec:intro}
Środowiska programistyczne ułatwiają i przyśpieszają pisanie kodu m.in. poprzez proponowanie 
słów kluczowych oraz nazw zdefiniowanych w programie. Dzięki tej funkcji programista nie musi pisać ręcznie 
w całości długiej nazwy zmiennej lub metody, a w niektórych przypadkach nie musi pisać jej wcale. 
% Przykładem takiego zachowania jest zaproponowanie słowa kluczowego 'except' po słowie 'try' w języku Python,
% lub propozycja nazwy zmiennej poprzez przechowywanie wszystkich zmiennych w słowniku.  
Podejście to charakteryzuję się parsowaniem na bieżąco kodu programu, oraz na podstawie reguł rządzących danym 
językiem programowania, proponowania nazw znajdujących sie w jego drzewie rozkładu, lub zaproponowaniu któregoś 
ze słów kluczowych na podstawie wcześniejszych ich wystąpień. Przykładem takiego zachowania może być zaproponowanie 
bloku \begin{math}else\end{math} po bloku \begin{math}if\end{math}. 
Jednak takie podejście wiążę się z wieloma wadami.
\begin{itemize}
	\item Stworzenie takiego systemu uzupełniającego wiąże się ze zdefiniowaniem wielu skomplikowanych
	reguł, które różnią się dla każdego języka programowania. 
	\item System generujący propozycje nie uwzględnia kontekstu pisanego kodu. Na przykład w kodzie 
	aplikacji internetowej można zaobserwować wiele powtarzających się szablonów, które będą się różnić 
	od szablonów występujących w kodzie jądra systemu operacyjnego. 
	\item Taki system ma problem z ocenieniem, która podpowiedź ma największe prawdopodobieństwo
	pojawienia się, przez co często podaje je, w z góry założonej kolejności np. posortowane leksykograficznie. 
\end{itemize}
\subsection{Problem}
\begin{description}
\item[Problem]
\hfill\\
Istnieje wiele rodzajów uzupełniania kodu: 
\begin{itemize}
	\item przewidzenie kolejnego słowa (tokenu),
	\item przewidzenie dłuższej sekwencji słów (na przykład dokończenie linijki),
	\item generowanie funkcji, na podstawie jej opisu w komentarzu,
	\item uzupełnianie brakujących linii lub tokenów, w zaznaczonych miejscach w kodzie.
\end{itemize}
W tej pracy skupiam się na pierwszym z rodzajów tego problemu. Moim celem jest stworzenie systemu, który rozwiąże 
wcześniej wymienione problemy, oraz zaimplementowanie go jako wtyczka do środowiska SublimeText. \\

\item[Kod a język naturalny]
\hfill\\
\label{similarities}
Metody statystyczne oraz rekurencyjne sieci neuronowe mają swoje zastosowanie w bardzo dużej ilości dziedzin, m.in. 
predykcji, klasyfikacji lub filtracji sygnałów. Rozwiązywany przeze mnie problem jest specjalnym przypadkiem problemu 
klasyfikacji, który opiszę w dalszej części pracy. Z tego względu zdecydowałem się właśnie na zastosowanie ich w moim rozwiązaniu.  

Języki programowania dzielą wiele cech wspólnych z językiem naturalnym.
Jednym z zastosowań języka naturalnego jest opisywanie algorytmów w skończonej liczbie kroków, co jest również jedynym zastosowaniem 
języków programowania. Logika stojąca za wyrażaniem kolejnych kroków jest taka sama. Oba typy języków używane są 
do komunikacji, naturalny używany pomiędzy ludźmi, natomiast programowania między człowiekiem a komputerem. 
Jednak najważniejszą łączącą je cechą jest ich powtarzalność. W obu z dużym prawdopodobieństwem po jednym 
słowie może wystąpić tyko względnie niewielki zbiór innych słów. 

Istotną różnicą dzielącą te rodzaje języków jest możliwość nadawania dowolnych nazw obiektom oraz metodom w językach programowania. 
Powoduje to, że nie można objąć wszystkich slów w słowniku danych treningowych. Słowa tego typu nazywane są słowami poza 
słownikiem. Różnica ta jest na tyle znacząca, że powoduje konieczność wprowadzenia zmian adaptacyjnych w 
metodach dotyczących modelowania języka naturalnego. \\

% \subsection {Wyzwania}
% \label{chellenges}
% Głównym wyzwaniem oraz detalem różniącym języki programowania od języków naturalnych, jest możliwość nadawania dowolnych
% nazw obiektom oraz metodom, przez co nie można objąć wszystkich słów w słowniku danych treningowych. Słowa tego typu 
% nazywane są słowami poza słownikiem. Zwiększanie wielkości słownika nigdy nie obejmie wszystkich możliwych nazw, natomiast 
% bardzo spowolni ostatni krok algorytmu, którym jest obliczenie wyznaczenie funkcji softmax. Jak zostało pokazane w publikacji 
% \cite{hellendoorn} od pewnego momentu większy rozmiar słownika zaczyna wpływać negatywnie na skuteczność modelu. 

% Nadmierne dopasowanie modelu do danych treningowych może wystąpić przy zbyt długim treningu. Taki model 
% zacznie dawać bardzo dobre predykcje na zbiorze treningowym jednak bardzo słabo poradzi sobie na zbiorze 
% walidacyjnym. Zamiast zgeneralizować problem model nauczy się danych treningowych 'na pamięć'. 

% Przewidywanie kilku tokenów w przód. Omawiany w tej pracy model, jest w stanie wykonywać kilka predykcji w przód,  
% jednak znacząco utrudnia to zadanie inżynierskie oraz miałoby negatywny wpływ na korzystanie ze wtyczki 
% w warunkach rzeczywistych. Całkowita skuteczność modelu o skuteczności wynoszącej na przykład {70\%} dla pojedyńczych 
% tokenów przy próbie przewidzenia 3 tokenów w przód spadłaby do \begin{math}0.7^3 = 0.343\end{math}, co było by nieakceptowalne 
% w warunkach rzeczywistych. 

\item [Zastosowania]
\hfill\\
\label{zastosowania}
Główny zastosowaniem tworzonego systemu jest usprawnienie pracy programisty. Jednak przy założeniu, że model działa dobrze 
istnieje więcej przypadków użycia: 
\begin{itemize}
	\item Tworzenie kodu na urządzeniu mobilnym. W dzisiejszych czasach urządzenia mobilne mają ogromne możliwości. 
	Jedyną rzeczą, która je powstrzymuje przed użyciem ich w celu rozwoju oprogramowania, jest mała klawiatura dotykowa nie
	udostępniająca szybkiego dostępu do znaków specjalnych. Wtyczka mogłaby znacznie usprawnić pisanie poprzez przewidywanie znaków 
	specjalnych (z czym jak pokażę później radzi sobie bardzo dobrze), jak i długich, niewygodnych do napisania nazw występujących w kodzie.

	\item Szukanie błędów w kodzie. Model może obliczyć prawdopodobieństwo wystąpienie następnego tokenu po czym sprawdzić czy 
	pokrywa się on z faktycznie występującym tokenem. W ten sposób możemy określić miejsce w kodzie w którym należy 
	spodziewać się, że został popełniony błąd. 

	\item Kompresja kodu. Modele Sequence2Sequence sprawdzają się w zadaniu kompresji. Model mógłby nauczyć się wygenerować resztę programu 
	na podstawie kilku pierwszych tokenów. W ten sposób, zamiast zapisywać cały kod źródłowy moglibyśmy zapamiętywać jedynie kilka 
	krótkich sekwencji. 
\end{itemize}
\end{description}



\subsection {Cel}
Celem tej pracy jest stworzenie modelu uczenia maszynowego przewidującego kolejny token podczas pisania kodu programu w języku python, 
oraz implementacja go jako wtyczki do zintegrowanego środowiska programistycznego SublimeText3. 
         % Wygodnie jest trzymać każdy rozdział w osobnym pliku.
\newpage % Rozdziały zaczynamy od nowej strony.
\section{Podłoże pracy}
W ciągu ostatnich kilku lat przetwarzanie języka naturalnego bardzo się rozwinęło. Wraz z kolejnymi 
badaniami udało się uzyskać coraz lepsze rezultaty. Jednak dział badający zachowanie tych modeli 
na językach programowania jest nowy, co możemy zaobserwować po zbiorze prac \cite{ml4code}
z nim związanych. Jest w nim dużo miejsca na nowe podejścia oraz badania. W tym rozdziale omówię 
opiszę prace istotne lub podobne do mojej.  


\subsection{Sieci neuronowe w przewidywaniu języków programowania}
Subhasis Das, Chinmayee Shah \cite{contextual_code_completion} porównują ze sobą modele
\begin{itemize}
    \item model z wagami o stałej długości okna (fixed window weight model)
    \item model macierzy wektorów (matrix vector model)
    \item sieć neuronowa z wyprzedzeniem (feed-forward neural network)
    \item model z wyprzedzeniem oraz miękka uwagą (feed-forward model with soft attention)
    \item model rekurencyjny z warstwą GRU
\end{itemize}
Kod wejściowy jest poddany tokenizacji przy pomocy wyrażeń regularnych oraz oceniane na podstawie dokładnego 
dopasowania pierwszego przewidzianego tokenu oraz na podstawie 3 najlepszych sugestii. Do testów 
używane są kody bibliotek Django, Twisted oraz jądra systemu Linux. Połowa plików źródłowych jednego z 
projektów używana jest jako zbiór treningowy natomiast druga połowa jako zbiór walidacyjny. Wszystkie modele
osiągają dokładność przewidzeń równą około 80\% dla 3 najlepszych sugestii, najlepiej radzi sobie model miękka uwagą
z dokładnością 83.6\%. Problem słów poza słownikiem rozwiązany jest przy pomocy słownika przypisującego 
token o nieznanej wartości do słowa które wpisał użytkownik. Moja praca różni się tym, że skupiam się 
wyłącznie na sieciach rekurencyjnych, jako że w powyższej warstwa LSTM nie została uwzględniona. Próbuję 
również uogólnić przewidywany kod poprzez trening na projektach zawierających różne biblioteki aby sprawić
by wtyczka była użyteczna w praktyce. \\

Hellendoorn i Devanbu przeprowadzili eksperyment polegający na wykonaniu 15000 predykcji dla 
środowiska Visual Studio.Jako zbioru danych używają 14000 plików źródłowych w języku Java. 
W swojej pracy porównują skuteczność modelu n-gram z modelami rekurencyjnymi. 
Prezentują również dynamicznie aktualizowane modele n-gram działające w zagnieżdżonym zasięgu, rozwiązując 
w ten sposób problem skończonego słownika oraz znacznie usprawniając sugestie. Rezultaty tej pracy 
pokazują, że pomimo znacznie lepszych wyników sieci rekurencyjnych w zadaniu modelowania języka 
naturalnego, modele n-gram w niektórych przypadkach radzą sobie lepiej z przewidywaniem kodu. Jednym
z przytoczonych przykładów są metody wbudowane w język, dla których głębokie sieci działają lepiej, 
jednak przegrywają przy często występujących, zróżnicowanych,  mało popularnych bibliotekach zewnętrznych, 
w których model n-gram naturalnie radzi sobie lepiej, jednak przegrywa pod innymi względami. Swoje 
eksperymenty przeprowadzają dla stałych wartości hiperparametrów, w swojej pracy chcę również
zająć się strojeniem wybranych modeli. 


\subsection {Modele statystyczne w przewidywaniu języków programowania}
Myroslava Romaniuk \cite{pharo} pokazuje, że same modele statystyczne w tym przypadku n-gram, dokładnie 
unigram oraz bigram radzą sobie 
z automatycznym uzupełnianiem kodu. W swojej pracy usprawnia działanie wtyczki do środowiska programistycznego 
Pharo. Zaproponowany model trenuje na 50 projektach w tym języku osiągając dokładność około 40\%. 
W swojej pracy łącze modele rekurencyjne właśnie z połączonymi modelami unigram oraz bigram.

\subsection {Rozwiązania komercyjne}
W dużej mierze do powstania tej pracy przyczyniły się istniejące już rozwiązania komercyjne. Niestety 
ze względów licencyjnych nie są ujawnione dokładnie mechanizmy stojące za ich działaniem, metody użyte 
do treningu oraz dokładna skuteczność, przez co są słabym punktem odniesienia w porównywaniu odniesionych 
wyników. \\

Tabnine \cite{tabnine} jest wtyczką do najpopularniejszych środowisk programistycznych realizującą predykcję kolejnego tokenu w 
większości stosowanych języków programowania. Do jej treningu zostało wykorzystane 2 miliony projektów ze strony github \cite{github}. 
Wtyczka opiera się na GPT-2, które używa architektury transformerów. W jej skład wchodzą również zaimplementowane przez twórców
reguły dotyczące języka. Podejście to jest bardzo nowatorskie przez wykorzystanie jeszcze nie zbadanych dokładnie modeli oraz 
różni się od przedstawionego w tej pracy.\\

Open AI realizuje generowanie kodu na podstawie opisu jego działania w komentarzu. Jest to połączenie zadania zrozumienia 
języka naturalnego przez maszynę z zadaniem klasyfikacji. Słownik zamiast składać sie z pojedyńczych tokenów skłąda sie z całych funkcji 
a na wejściu modelu otrzymujemy sekwencje słów zamiast poprzedzający kod. 
    % Umożliwia to również łatwą migrację do nowej wersji szablonu:
\newpage % Rozdziały zaczynamy od nowej strony.
 

\section{Użyte metody}
W tym rozdziale opiszę użyte przez siebie metody prowadzące, od zbioru kodów źródłowych programów, do działającego modelu 
przewidującej kolejny token w programie. \\\\\\ 

\subsection{Badane modele}
Rekurencyjne sieci neuronowe należą do rodziny sieci służących do przetwarzania sekwencji danych o określonej długości. Jak pokazali w swojej publikacji 
autorzy \cite{lstmvsgru}, sieć LSTM osiąga znacznie lepsze wyniki od sieci rekurencyjnych Hopfielda \cite{hopfield} oraz minimalnie lepsze wyniki 
od sieci GRU, kosztem dłuższego czasu treningu. Z tego względu w swoich badaniach skupiam się głównie na warstwie LSTM oraz w mniejszej mierze na warstwie GRU, 
która również wyprzedza sieć Hopfielda pod względem skuteczności. 

Wszystkie eksperymenty badające wpływ hiperparametrów przeprowadzę na sieci LSTM, oraz najlepszą ich kombinację 
zbadam przy wykorzystaniu warstwy GRU w celu porównania ich skuteczności w zadaniu przewidywaniu kodu. Na koniec zbadam również zachowanie wyznaczonego modelu 
dla różnych rozmiarów słownika. 

Jak pokazał w swojej pracy Yoon Kim \cite{kim} połączenie warstwy rekurencyjnej z warstwą zanurzeń poprzez warstwę splotową może mieć pozytywny wpływ na skuteczność 
modelu. Jednak badanie to wykonywał jedynie w zadaniu modelowania języka naturalnego oraz dla modeli opartych na pojedyńczych znakach. Jednym z wykonanych przeze mnie 
eksperymentów będzie zastosowanie tej techniki w moim zadaniu, oraz oceny skuteczności tego podejścia dla modeli opartych na tokenach. 

Ogólny badany przeze mnie model uczenia głębokiego został przedstawiony na rysunku \ref{fig:architektura}. Składa się on z sieci zanurzeń, opcjonalnej warstwy CNN, 1 lub 2 
warstw rekurencyjnych LSTM, oraz warstwy klasyfikującej.

Na architekturę całej wtyczki składa się wybrany przeze mnie model uczenia głębokiego połączony z 2 modelami typu n-gram. Zasada działania tej kombinacji 
została opisana w rozdziale \ref{oov}. 

\begin{figure}[!h]
	% Znacznik \caption oprócz podpisu służy również do wygenerowania numeru obrazka;
	\caption{Ogólny badany model. Opcjonalne warstwy zaznaczone linią przerywaną}
	% dlatego zawsze pamiętaj używać najpierw \caption, a potem \label
    \label{fig:architektura}
    % Zamiast width można też użyć height, etc. 
    \centering \includegraphics[width=160mm, height=25mm]{architektura.png}
\end{figure}


\begin{table}[ht]
    \centering
    \resizebox{\textwidth}{!}{\begin{tabular}{ccccc}
            \hline
            \multicolumn{1}{|c|}{Długość sekwencji}   & \multicolumn{1}{c|}{Warstwa CNN}          & \multicolumn{1}{c|}{Liczba warstw LSTM} & \multicolumn{1}{c|}{Liczba neuronów w warstwie} & \multicolumn{1}{c|}{Liczba wszystkich wag} \\ \hline
            \multicolumn{1}{|c|}{1}                   & \multicolumn{1}{c|}{nie}                  & \multicolumn{1}{c|}{1}                  & \multicolumn{1}{c|}{512}                        & \multicolumn{1}{r|}{12,016,705}                         \\ \hline
            \multicolumn{1}{|c|}{\multirow{5}{*}{5}}  & \multicolumn{1}{c|}{\multirow{4}{*}{nie}} & \multicolumn{1}{c|}{\multirow{3}{*}{1}} & \multicolumn{1}{c|}{128}                        & \multicolumn{1}{r|}{3,302,593}                         \\ \cline{4-5} 
            \multicolumn{1}{|c|}{}                    & \multicolumn{1}{c|}{}                     & \multicolumn{1}{c|}{}                   & \multicolumn{1}{c|}{256}                        & \multicolumn{1}{r|}{6,076,225}                         \\ \cline{4-5} 
            \multicolumn{1}{|c|}{}                    & \multicolumn{1}{c|}{}                     & \multicolumn{1}{c|}{}                   & \multicolumn{1}{c|}{512}                        & \multicolumn{1}{r|}{12,016,705}                         \\ \cline{3-5} 
            \multicolumn{1}{|c|}{}                    & \multicolumn{1}{c|}{}                     & \multicolumn{1}{c|}{2}                  & \multicolumn{1}{c|}{128}                        & \multicolumn{1}{r|}{3,434,177}                         \\ \cline{2-5} 
            \multicolumn{1}{|c|}{}                    & \multicolumn{1}{c|}{tak}                  & \multicolumn{1}{c|}{1}                  & \multicolumn{1}{c|}{128}                        & \multicolumn{1}{r|}{3,303,649}                         \\ \hline
            \multicolumn{1}{|c|}{\multirow{5}{*}{10}} & \multicolumn{1}{c|}{\multirow{4}{*}{nie}} & \multicolumn{1}{c|}{\multirow{2}{*}{1}} & \multicolumn{1}{c|}{128}                        & \multicolumn{1}{r|}{3,302,593}                         \\ \cline{4-5} 
            \multicolumn{1}{|c|}{}                    & \multicolumn{1}{c|}{}                     & \multicolumn{1}{c|}{}                   & \multicolumn{1}{c|}{512}                        & \multicolumn{1}{r|}{12,016,705}                         \\ \cline{3-5} 
            \multicolumn{1}{|c|}{}                    & \multicolumn{1}{c|}{}                     & \multicolumn{1}{c|}{2}                  & \multicolumn{1}{c|}{128}                        & \multicolumn{1}{r|}{3,434,177}                         \\ \cline{3-5} 
            \multicolumn{1}{|c|}{}                    & \multicolumn{1}{c|}{}                     & \multicolumn{1}{c|}{2}                  & \multicolumn{1}{c|}{512}                        & \multicolumn{1}{r|}{14,115,905}                         \\ \cline{2-5} 
            \multicolumn{1}{|c|}{}                    & \multicolumn{1}{c|}{tak}                  & \multicolumn{1}{c|}{1}                  & \multicolumn{1}{c|}{128}                        & \multicolumn{1}{r|}{3,303,649}                         \\ \hline
            \multicolumn{1}{|c|}{\multirow{2}{*}{15}} & \multicolumn{1}{c|}{\multirow{2}{*}{nie}} & \multicolumn{1}{c|}{\multirow{2}{*}{1}} & \multicolumn{1}{c|}{128}                        & \multicolumn{1}{r|}{3,302,593}                         \\ \cline{4-5} 
            \multicolumn{1}{|c|}{}                    & \multicolumn{1}{c|}{}                     & \multicolumn{1}{c|}{}                   & \multicolumn{1}{c|}{512}                        & \multicolumn{1}{r|}{12,016,705}                         \\ \hline
            \end{tabular}}
    \caption{Zestawienie wykonywanych eksperymentów} 
    \label{eksperymenty}
\end{table} 
Zdecydowałem, że nie ma potrzeby testowania wszystkich możliwych kombinacji parametrów w tabeli, ponieważ zajęło by to bardzo dużo czasu, oraz już na podstawie prac 
\cite{hellendoorn, pythia} możemy łatwo stwierdzić, że część kombinacji nie osiągnie konkurencyjnych wyników przy pozostałych, na przykład pojedyńcza warstwa LSTM, o 
128 komórkach i długości okna równej \begin{math}1\end{math} jest zdecydowanie za prostym modelem aby przechwycić wszystkie zależności między danymi. Przeprowadzane 
przeze mnie eksperymenty przedstawione są w tabeli \ref{eksperymenty}. 

\subsection{Hiperparametry}
Badając różne w możliwości układów warstw należy rozważyć kilka szczególnych kwestii dotyczących hiperparametrów testowanych przeze mnie modeli. 
Poszczególnie zbadam 
\begin{itemize}
    \item dla jakiej długości sekwencji wejściowej model osiągnie najlepsze wyniki,
    \item jak rozmiar słownika wpływa na skuteczność modelu, 
    \item jaka liczba warstw rekurencyjnych osiągnie najlepsze wyniki,
    \item dla jakiej liczby neuronów w warstwie model osiągnie najlepsze wyniki, 
    \item jaki wpływ ma warstwa CNN, 
    \item który rodzaj warstw rekurencyjnych osiągnie najlepsze wyniki.
\end{itemize}
\subsection{Projektowanie wtyczki}
\begin{description}
\item[Wstępne przetwarzanie danych]
\hfill \\
Przed dostarczeniem danych do modelu zostają one odpowiednio przetworzone. Operacja ta polega na usunięciu wszystkich wykomentowanych linii kodu 
w celu zapobięgnięcia uczenia modelu na słowach nie będących kodem. Następnie usuwane są wszystkie puste linie gdyż nie niosą one żadnej informacji o 
kolejnych tokenach. Ostatnim krokiem jest usunięcie wcięć w kodzie, gdyż wynikają one ze struktury programów w języku python, przez co próba przewidywania  
ich przy pomocy uczenia maszynowego nie ma sensu. \\

\item[Modelowanie tokenów]
\hfill \\
Jak pokazują w swojej publikacji autorzy \cite{character-level} mimo tego, że modele budujące kolejne słowa poprzez 
przewidywanie pojedyńczego znaku (character-level model) radzą sobie dobrze z modelowaniem języka naturalnego, oraz rozwiązują problem rozmiaru słownika, 
działają znacznie gorzej z językami programowania. Wniosek ten również potwierdza w swojej pracy autor \cite{erik} porównując model przewidujący 
znaki z modelem przewidującym tokeny. Z tego powodu realizuję tylko modele oparte na tokenach (token-level model). 

Pierwszym krokiem jest zbudowanie słownika tokenów, które mogą pojawić się w kodzie. Buduję go poprzez przetworzenie wszystkich kodów źródłowych obu zbiorów treningowego oraz 
walidacyjnego modułem tokenize \cite{tokenize} wbudowanym w język Python. Moduł ten przyjmuje na wejściu kod źródłowy programu następnie zwraca listę kolejnych tokenów (nazw zmiennych, 
znaków specjalnych, słów kluczowych). Upewnia się on również czy kod jest poprawnie napisany, na przykład czy wszystkie nawiasy lub apostrofy zostały zamknięte. Pliki zawierające błędy w 
kodzie zostają pominięte. 
Znaki nowej lini również traktuję jako token, jednak nie uwzględniam wcięć w kodzie ze względu na to, że większość środowisk programistycznych stawia je 
automatycznie. Na przykład z kodu źródłowego: 
\begin{addmargin}[10mm]{0mm}
    \begin{lstlisting}[
        language=Python,
        numbers=left,
        firstnumber=1,
        caption={Przykładowy program Python},
        aboveskip=10pt
    ]
    for x in range(2, 10): 
        print("hello world")
    \end{lstlisting}
    \end{addmargin}
otrzymamy listę \textbf{ [for, x, in, range, (, 2, 10, ), :, \textbackslash n, print, (, "hello world", )]}.
Następnie sortuje wszystkie wygenerowane tokeny według częstości występowania oraz wybieram top-n tokenów jako słownik i każdemu z nich przypisuję unikalną liczbę naturalną. 
Utworzony w ten sposób słownik nie jest kompletny ponieważ nie obejmuje on wszystkich możliwych nazw występujących w kodzie. Takiego rodzaju tokeny
zostają zastąpione sztucznym tokenem '<UNKNOWN>'. W głównej mierze są to unikalne nazwy zmiennych oraz ciągi znaków. Zastępowanie tokenu '<UNKNOWN>' prawdziwym tokenem omawiam w 
rozdziale \ref{oov}\\

\item[Wybór podzbioru danych]
\hfill \\ 
Jak już wspomniałem w sekcji \ref{sec:dataset-background} dotyczącej zbioru danych, trening odbywa sie na podzbiorze wszystkich zgromadzonych danych. W tym celu wybrałem 9 najpopularniejszych, 
zewnętrznych bibliotek języka Python, stosowanych w zróżnicowanych dziedzinach rozwoju oprogramowania. Poniżej zamieszczam listę wybranych bibliotek, wraz z krótkim opisem:
\begin{itemize}
    \item Django - rozwój aplikacji sieciowych,
    \item Numpy - wykonywanie obliczeń matematycznych wysokiego poziomu,
    \item Requests - wysyłanie zapytań http, 
    \item Flask - rozwój aplikacji sieciowych,
    \item TensorFlow - Głębokie uczenie maszynowe,
    \item Keras - Wysokopoziomowe uczenie maszynowe,
    \item PyTorch - Głębokie uczenie maszynowe, 
    \item Pandas - Zarządzanie dużymi zbiorami danych, 
    \item PyQt - Tworzenie interfejsów użytkownika.
\end{itemize} 
Aby upewnić się, że wybrany przeze mnie zbiór danych poprawnie oddaje rzeczywistość porównałem liczbę plików źródłowych zwierających konkretną bibliotekę ze zbioru, z odpowiadającą 
liczbą wszystkich plików źródłowych na platformie GitHub \cite{github}. Stosunek tych dwóch liczb został przedstawiony w kolumnie 'Stosunek'. Zestawienie to znajduje się w tabeli \ref{tab:dataset-compare} 
\begin{table}[!h] \centering
    % Znacznik \caption oprócz podpisu służy również do wygenerowania numeru tabeli;
    \caption{Zestawienie zbioru danych z platformą GitHub}
    % dlatego zawsze pamiętaj używać najpierw \caption, a potem \label.
    \label{tab:dataset-compare}
    
    \begin{tabular} {| c | c | r | r |} \hline
        Biblioteka & Liczba plików GitHub & Liczba plików w zbiorze danych & Stosunek\\\hline\hline
        Django & 187000000 & 26732 & 0.014\% \\\hline
        Numpy & 55000000 & 9058 & 0.016\% \\ \hline
        Requests & 42000000 & 6339 & 0.015\%\\ \hline
        Pandas & 19000000 & 1328 & 0.007\%\\ \hline
        Flask & 17000000 & 3230 & 0.019\% \\ \hline
        TensorFlow & 10000000 & 96 & 0.001\%\\ \hline
        Keras & 3000000& 72 & 0.002\%\\ \hline
        PyQt & 1000000 & 132 & 0.013\%\\ \hline
        PyTorch & 1000000 & 0 & 0\%\\ \hline
    \end{tabular}
\end{table}

Jak możemy zaobserwować stosunek liczby plików jest dosyć zbliżony dla większości wybranych bibliotek wynosi około 0.015\%. Wyjątkami są 
biblioteki uczenia maszynowego. Może to wynikać z tego, że dane pochodzą z 2018 roku. Jest to czas, w którym istniały początkowe wersje tych bibliotek oraz dopiero
zaczynały zyskiwać na popularności. Od tego czasu również biblioteka Keras została scalona z biblioteką TensorFlow co znacznie wpłynęło na jej popularność w dzisiejszych czasach. 

Uznaję, że dane w wystarczająco dobrym stopniu oddają częstotliwość zastosowania bibliotek. Końcowy podzbiór stworzyłem poprzez wybranie 1\% plików źródłowych z każdej biblioteki. 
Końcowe zestawienie znajduje się w tabeli \ref{tab:dataset-compare-github}

\begin{table}[!h] \centering
    % Znacznik \caption oprócz podpisu służy również do wygenerowania numeru tabeli;
    \caption{Zestawienie zbioru i podzbioru danych}
    % dlatego zawsze pamiętaj używać najpierw \caption, a potem \label.
    \label{tab:dataset-compare-github}
    
    \begin{tabular} {| c | c | r | r |} \hline
         & Cały zbiór danych & Podzbiór treningowy & Podzbiór walidacyjny \\\hline\hline
        Pliki & 150000 & 9103 & 4504 \\\hline
        Tokeny & 114641650 & 9118453 & 4482600 \\ \hline
    \end{tabular}
\end{table}
Rozmiar wykorzystanego zbioru danych różni się do zbioru użytego w publikacji \cite{hellendoorn} w którym użyto 16 milionów tokenów do treningu, oraz 
5 milionów tokenów w celach walidacji. Na różnicę tą zdecydowałem się ze względu na ograniczenia sprzętowe różniące oba projekty. \\


\item[Trening]
\hfill\\ 
Zadanie polega na przewidzeniu kolejnego tokenu na podstawie zadanej sekwencji tokenów. Długość sekwencji jest stała oraz wyrażona poprzez wielkość okna będącą jednym z badanych hiperparametrów. 
Dla każdego z tokenów model wylicza jego wektor zanurzenia, wykonuje jeden krok w sieci rekurencyjnej po czym stosuje warstwę klasyfikującą 
(Dense layer) w celu wygenerowania logitów wyrażających logistyczne-prawdopodobieństwo kolejnego tokenu. Zatem dla zadanego okna tokenów długości \begin{math}W\end{math}:
\begin{math}[t_1, t_2, ... t_W]\end{math} obliczam wynik dla każdego możliwego wyjścia \begin{math}j, s_j\end{math} jako funkcję z wektorów tokenów \begin{math}v_{t_i}\end{math}
z tokenów z okna.\\
\centerline{\begin{math}g = [g_1, g_2, ..., g_W]\end{math}}
\centerline{\begin{math}[g_1, g_2, ..., g_W] = RNN([v_{t_1}, v_{t_2}, ..., v_{t_W}])\end{math}}
\centerline{\begin{math}s_j=p_{j}^{T}[g]\end{math}}\\\\
Minimalizowana funkcja strat jest entropią krzyżową pomiędzy prawdopodobieństwami \begin{math}softmax\end{math} dla każdego możliwego wyjścia 
a wykonaną predykcją.Funkcja wyrażoną wzorem: \\\\
\centerline{\begin{math}L = -log(\frac{exp(s_{t_o})}{\sum_{j}exp(s_{t_j})})\end{math}}\\\\
gdzie \begin{math}t_o\end{math} jest zaobserwowanym tokenem wyjściowym a \begin{math}g_i\end{math} wyjściem \begin{math}i-tej\end{math} komórki
którejś z badanych sieci rekurencyjnych. 

Wagi sieci aktualizowane są po przetworzeniu porcji danych (mini-batch), której rozmiar jest stały. Przy treningach sieci rekurencyjnych rozmiar
ten jest jedną z wartości kluczowych dla dobrej wydajności sieci. W moich eksperymentach wynosi on 128. Jest to kompromis pomiędzy rozsądnym 
czasem treningu oraz jakością wyjściowych sugestii. Jest to również najczęściej wybierana wartość w przytoczonych przeze mnie publikacjach.

Każda testowana architektura trenowana jest przez 25 epok. Wartość tą wybrałem na podstawie własnych eksperymentów wstępnych, z których wynika, że 
powyżej tej liczby model nie osiągał już lepszych rezultatów. Zbyt długi trening może również doprowadzić to przetrenowania modelu, czego 
należy unikać. 
\end{description}
\subsection{Ewaluacja}
\label{evaluation}
Jako, że badane przeze mnie modele tworzone są z myślą użycia ich w postaci wtyczki do środowiska programistycznego nie ma sensu ocenianie ich na podstawie pierwszej, najlepszej predykcji. 
Zamiast tego użyję dwóch następujących metryk: 

\begin{description}
    \item[Ewaluacja bez uwzględnienia kolejności] 
    \hfill \\ Jeśli poszukiwane słowo znajduje się w pierwszych \begin{math}n\end{math} najlepszych predykcjach, bez znaczenia na którym miejscu uważam sugestię za poprawną. Metrykę tą zastosowano przy ocenianiu systemów
    Pythia \cite{pythia} dla której \begin{math}n = 5\end{math} oraz w pracy \cite{contextual_code_completion} gdzie \begin{math}n = 3\end{math}. Zastosowanie jej pozwoli na porównanie 
    wyników z tymi publikacjami. Miara ta jest dalej nazywana "Top n"\\
    \item[Ewaluacja z uwzględnieniem kolejności] 
    \hfill \\ Jednym z przytoczonym problemów we wstępie \ref{sec:intro} jest to, że wiele wtyczek nie uwzględnia kolejności sugestii oraz proponuje je na przykład posortowane leksykograficznie. Proponowane przeze mnie rozwiązanie 
    sortuje predykcje na podstawie prawdopodobieństwa ich wystąpienia. Należy uwzględnić tą kolejność w metryce. Ocena obliczana jest poprzez podzielenie prawdopodobieństwa poprawnej predykcji przez 
    jego indeks w zbiorze zebranych predykcji. Dla przykładu powiedzmy, że model proponujący 10 sugestii zostaje użyty do wykonania 4 predykcji. Pierwsza wystąpi na 1. miejscu, druga na 3. miejscu, trzecia nie zmieści w w 10 najlepszych, 
    a czwarta na 8. miejscu. W takim przypadku ocena modelu będzie wynosić \begin{math}(\frac{1}{1}+ \frac{1}{3}+ 0 +\frac{1}{8})/4 = 0.36\end{math}. W ten sposób wyższe predykcje oceniane są zdecydowanie lepiej.
    Przy ocenie użyję 10 najlepszych sugestii.  Metryki tej używa Erik van Scharrenburg \cite{erik} w swojej pracy. Zastosowanie jej pozwoli na porównanie wyników. Miara ta jest dalej 
    nazywana "Top n z kolejnością" \\

  \end{description}


\subsection{Szczegóły implementacji}
\label{oov}
\begin{description}
\item[Problem słów poza słownikiem]
\hfill \\
Największym wyzwaniem przy modelowaniu języka programowanie jest rozwiązanie problemu słów poza słownikiem opisanego w podrozdziale \ref{zastosowania}. 
Jednym z podejść zastosowanym przez autorów \cite{contextual_code_completion} jest zastępowanie tokenów nie występujących w słowniku 
specjalnymi tokenami pozycyjnymi. Tego typu tokenowi, który powtarza się więcej niż raz w sekwencji, zostaje przypisany indeks jego wystąpienia 
odpowiadający jego pierwszemu wystąpieniu. W przypadku gdy przewidziany token nie mieści się w słowniku ale pojawił się wcześniej w 
sekwencji zostaje zastąpiony wcześniej podaną nazwą. Rozwiązanie to sprawdza się bardzo dobrze dla zmiennych o tej samej nazwie, znajdujących 
sie blisko siebie, na przykład w pętli \begin{math}for\end{math} języka \begin{math}C++\end{math}: 
\begin{addmargin}[10mm]{0mm}
    \begin{lstlisting}[
        language=C++,
        numbers=left,
        firstnumber=1,
        caption={Przeparsowana pętla for},
        aboveskip=10pt
    ]
    for (int POS_TOKEN_01=0; POS_TOKEN_01<10; POS_TOKEN_01++)
    \end{lstlisting}
    \end{addmargin}
Metoda ta jednak ogranicza się do długości badanej sekwencji, która jest bardzo krótka względem przeciętnej długości kodów programów.


Inną metodą jest zastosowanie warstwy splotowej zamiast warstwy zanurzeń zaproponowaną przez Yoon Kim \cite{kim}. Metoda ta osiąga skuteczność 
na poziomie najlepszych dotychczas znanych modeli, jednak zastosowana jest dla modeli języka naturalnego opartego na znakach, przez co 
prawdopodobnie nie sprawdziłaby się przy wybranych przeze mnie założeniach. 

Hellendoorn wraz z Devanbu \cite{hellendoorn} proponują połączenie sieci rekurencyjnych z modelami statystycznymi \begin{math}N-gram\end{math}. Metoda ta pozwala na bardziej 
ogólne predykcje. Pokazują również, że modele \begin{math}N-gram\end{math} radzą sobie lepie z przewidywaniem rzadko występujących unikalnych tokenów od sieci neuronowych, 
oraz kombinacja tych modeli osiąga mniejszą entropię niż każdy z tych modeli osobno. 

Swoje eksperymenty przeprowadzam właśnie z zastosowaniem kombinacji sieci rekurencyjnej z modelami unigram oraz bigram, preferując podpowiedzi wykonane przez bigram. Zastosowanie 
większych modeli \begin{math}N-gram\end{math} nie ma sensu, ponieważ nałożyłoby to dodatkowe koszty obliczeniowe spowalniając wykonywanie predykcji, a generowany przez nie zbiór 
byłby w większości przypadków pusty. 
Stosuje tę metodę poprzez zastąpienie słów nie występujące w słowniku tokenem \begin{math}<UNKNOWN>\end{math} oraz uczę model uczenia głębokiego przewidywać go w odpowiednich miejscach. Następnie jeśli pośród 
zebranych predykcji znajdzie sie token \begin{math}<UNKNOWN>\end{math} zastępuje go zbiorem będącym sumą zbiorów predykcji unigramu oraz bigramu. Oba modele typu n-gram uczone są na bieżąco,
na pisanym przez programistę tekście, zatem znają wszystkie użyte przez niego nazwy. \\

\item[Rozmiar Słownika]
\hfill \\
W eksperymentach z tabeli \ref{eksperymenty} stosuję stałą wartość rozmiaru słownika wynoszącą \begin{math}20000\end{math}. Odpowiada ona usunięciu słów które nie pojawiają sie w 
zbiorze danych więcej niż \begin{math}18\end{math} razy. Odcięte zostają w głównej mierze nazwy zdefiniowanych funkcji, zmiennych oraz ciągi znaków.
Wybór ten spowodowany jest tym, że wartość ta ma ogromny wpływ na czas treningu oraz rozmiar modelu zapisanego na dysku. 
Wydłużony czas treningu spowodowany jest koniecznością obliczenia wartości funkcji \begin{math}softmax\end{math} dla każdego ze słów, natomiast dużo większy rozmiar wynika z konieczności 
przeskalowania warstwy wejściowej oraz wyjściowej. Rozmiar ten różni się od rozmiarów wybranych w przytoczonych przeze mnie pracach które wynoszą odpowiednio 74064 \cite{hellendoorn} oraz 
2000 \cite{contextual_code_completion}. W celu odpowiedzi na pytanie czy rozmiar słownika ma znaczenie na badanie modelu przeprowadzę jeden eksperyment polegający na treningu modelu o najlepszej 
skuteczności na rozmiarze słownika 74064 zaproponowanego przez Hellendoorn'a \cite{hellendoorn}. 
Przykłady odciętych tokenów w słowniku rozmiaru 20000:\\ \begin{math}accept\_inplace, IVGMM, book\_names, allvars, parent\_cards, 'references'\end{math}.\\
\item[Wektory zanurzenia]
\hfill \\
Wektory zanurzeń odpowiedzialne są za przypisanie każdemu ze słów wektora, którym możemy wyrazić prawdopodobieństwo tokenu jako rezultat wielowarstwowej sieci neuronowej na wektorach 
o ograniczonej liczbie tokenów sąsiednich. W mojej pracy realizowane są one poprzez warstwę \begin{math}Embedding\end{math} należącą do biblioteki TensorFlow, uczącą się równolegle z całym modelem. 
Wymiary wektorów zanurzeń są stałe dla każdego z wykonywanych eksperymentów oraz wynoszą one \begin{math}32\end{math}. Zdecydowałem się na mniejszy wymiar wektorów niż przedstawiony 
w pracy \cite{hellendoorn}, która wynosiła 128 z powodu, że używam mniejszego słownika przez co znajduję się w nim dużo mniej zależności, które można wyrazić mniejszym wymiarem.\\
\item[Optymalizator]
\hfill \\
W treningu używam optymalizatora \begin{math}Adam\end{math} o domyślnych parametrach dla każdego z modeli.
Wybór ten wynikał z tego, że celem eksperymentów było badanie różnic wynikających z użycia odmiennych architektur. 
Odpowiednie strojenie optymalizatora typu SGD było by bardzo czasochłonne oraz komplikowałoby porównywanie ze sobą testowanych modeli.
 Jest to również optymalizator używany w pracach 
z którymi porównam uzyskane przeze mnie wyniki.  
\end{description}

 % wystarczy podmienić swoje pliki main.tex i eiti-thesis.cls
                            % na nowe wersje, a cały tekst pracy pozostaje nienaruszony.
\newpage % Rozdziały zaczynamy od nowej strony.
 

\section{Analiza przeprowadzonych eksperymentów}
\subsection{Architektury}
\begin{description}

\item[Zbadane architektury]
\hfill\\
\begin{table}[ht]
    \centering
    \resizebox{\textwidth}{!}{\begin{tabular}{cccclll}
        \hline
        \multicolumn{1}{|c|}{Długość sekwencji}   & \multicolumn{1}{c|}{Warstwa CNN}          & \multicolumn{1}{c|}{Liczba warstw LSTM} & \multicolumn{1}{c|}{Liczba neuronów w warstwie} & \multicolumn{1}{l|}{Top 1} & \multicolumn{1}{l|}{Top 5} & \multicolumn{1}{l|}{Top 10 z kolejnością} \\ \hline
        \multicolumn{1}{|c|}{1}                   & \multicolumn{1}{c|}{nie}                  & \multicolumn{1}{c|}{1}                  & \multicolumn{1}{c|}{512}                        & \multicolumn{1}{l|}{0.27}     & \multicolumn{1}{l|}{0.51}     & \multicolumn{1}{l|}{0.41}                    \\ \hline
        \multicolumn{1}{|c|}{\multirow{5}{*}{5}}  & \multicolumn{1}{c|}{\multirow{4}{*}{nie}} & \multicolumn{1}{c|}{\multirow{3}{*}{1}} & \multicolumn{1}{c|}{128}                        & \multicolumn{1}{l|}{0.23}     & \multicolumn{1}{l|}{0.59}     & \multicolumn{1}{l|}{0.37}                    \\ \cline{4-7} 
        \multicolumn{1}{|c|}{}                    & \multicolumn{1}{c|}{}                     & \multicolumn{1}{c|}{}                   & \multicolumn{1}{c|}{256}                        & \multicolumn{1}{l|}{0.16}     & \multicolumn{1}{l|}{0.42}     & \multicolumn{1}{l|}{0.27}                    \\ \cline{4-7} 
        \multicolumn{1}{|c|}{}                    & \multicolumn{1}{c|}{}                     & \multicolumn{1}{c|}{}                   & \multicolumn{1}{c|}{512}                        & \multicolumn{1}{l|}{0.30}     & \multicolumn{1}{l|}{0.63}     & \multicolumn{1}{l|}{0.43}                    \\ \cline{3-7} 
        \multicolumn{1}{|c|}{}                    & \multicolumn{1}{c|}{}                     & \multicolumn{1}{c|}{2}                  & \multicolumn{1}{c|}{128}                        & \multicolumn{1}{l|}{0.27}     & \multicolumn{1}{l|}{0.62}     & \multicolumn{1}{l|}{0.41}                    \\ \cline{2-7} 
        \multicolumn{1}{|c|}{}                    & \multicolumn{1}{c|}{tak}                  & \multicolumn{1}{c|}{1}                  & \multicolumn{1}{c|}{128}                        & \multicolumn{1}{l|}{0.24}     & \multicolumn{1}{l|}{0.53}     & \multicolumn{1}{l|}{0.36}                    \\ \hline
        \multicolumn{1}{|c|}{\multirow{5}{*}{10}} & \multicolumn{1}{c|}{\multirow{4}{*}{nie}} & \multicolumn{1}{c|}{\multirow{2}{*}{1}} & \multicolumn{1}{c|}{128}                        & \multicolumn{1}{l|}{0.06}     & \multicolumn{1}{l|}{0.25}     & \multicolumn{1}{l|}{0.13}                    \\ \cline{4-7} 
        \multicolumn{1}{|c|}{}                    & \multicolumn{1}{c|}{}                     & \multicolumn{1}{c|}{}                   & \multicolumn{1}{c|}{512}                        & \multicolumn{1}{l|}{0.13}     & \multicolumn{1}{l|}{0.45}     & \multicolumn{1}{l|}{0.26}                    \\ \cline{3-7} 
        \multicolumn{1}{|c|}{}                    & \multicolumn{1}{c|}{}                     & \multicolumn{1}{c|}{2}                  & \multicolumn{1}{c|}{128}                        & \multicolumn{1}{l|}{0.34}     & \multicolumn{1}{l|}{0.66}    & \multicolumn{1}{l|}{0.47}                    \\ \cline{3-7} 
        \multicolumn{1}{|c|}{}                    & \multicolumn{1}{c|}{}                     & \multicolumn{1}{c|}{2}                  & \multicolumn{1}{c|}{512}                        & \multicolumn{1}{l|}{0.12}     & \multicolumn{1}{l|}{0.44}     & \multicolumn{1}{l|}{0.24}                    \\ \cline{2-7} 
        \multicolumn{1}{|c|}{}                    & \multicolumn{1}{c|}{tak}                  & \multicolumn{1}{c|}{1}                  & \multicolumn{1}{c|}{128}                        & \multicolumn{1}{l|}{0.17}     & \multicolumn{1}{l|}{0.56}     & \multicolumn{1}{l|}{0.32}                    \\ \hline
        \multicolumn{1}{|c|}{\multirow{2}{*}{15}} & \multicolumn{1}{c|}{\multirow{2}{*}{nie}} & \multicolumn{1}{c|}{\multirow{2}{*}{1}} & \multicolumn{1}{c|}{128}                        & \multicolumn{1}{l|}{0.03}     & \multicolumn{1}{l|}{0.10}     & \multicolumn{1}{l|}{0.06}                    \\ \cline{4-7} 
        \multicolumn{1}{|c|}{}                    & \multicolumn{1}{c|}{}                     & \multicolumn{1}{c|}{}                   & \multicolumn{1}{c|}{512}                        & \multicolumn{1}{l|}{0.10}     & \multicolumn{1}{l|}{0.38}     & \multicolumn{1}{l|}{0.21}                    \\ \hline
                                                  &                                           &                                         &                                                 &                            &                            &                                          
        \end{tabular}}
    \caption{Uzyskane skuteczności w eksperymentach} 
    \label{wyniki}
\end{table} 

\item[Wyróżniona architektura]
\hfill\\
\label{wyrozniony}
Ze wstępnych badań najlepsze wyniki uzyskał model o parametrach przedstawionych w tabeli \ref{parametry wstepne}. 
\begin{table}[ht]
    \centering
    \begin{tabular}{|l|l|}
        \hline
        \multicolumn{1}{|c|}{parametr} & wartość   \\ \hline
        liczba warstw rnn              & 2                 \\ \hline
        rodzaj rnn                     & lstm              \\ \hline
        liczba neuronów w warstwie     & 128               \\ \hline
        długość sekwencji wejściowej   & 10                \\ \hline
        liczba trenowanych parametrów  & 3,434,177         \\ \hline
        rozmiar słownika               & 20000             \\ \hline
        rozmiar wektora zanurzeń       & 32                \\ \hline
        rozmiar porcji danych          & 128               \\ \hline
        liczba epok                    & 25                \\ \hline
        optymalizator                  & adam              \\ \hline
        funkcja celu                   & entropia krzyżowa \\ \hline
        \end{tabular}
    \caption{Parametry najlepszego modelu z badań wstępnych} 

    \label{parametry wstepne}
\end{table} \\\\\\\\\\\\\\\\\\\\\\
Uzyskując skuteczności przedstawione w tabeli \ref{wyniki_najlepszego}. 
\begin{table}[ht]
    \centering
    \begin{tabular}{|l|l|}
        \hline
        \multicolumn{1}{|c|}{Miara}  & Wynik\\ \hline
        Top 1                            & 0.34  \\ \hline
        Top 3                            & 0.56  \\ \hline
        Top 5                            & 0.66  \\ \hline
        Top 5 z kolejnością              & 0.46  \\ \hline
        Top 10 z kolejnością             & 0.47  \\ \hline
        najmniejsza wartość funkcji celu & 1.67  \\ \hline
        \end{tabular}
    \caption{Wyniki uzyskane przez najlepszy model z badań wstępnych} 

    \label{wyniki_najlepszego}
\end{table} \\
Model ten osiągnął znacznie lepsze wyniki niż pozostałe badane modele. Na tym modelu zostaną wykonane dalsze badania 
dotyczące większego rozmiaru słownika oraz zastosowania warstwy gru zamiast lstm. 
\end{description}
\subsection{Różne rozmiary słownika}
Kolejnym etapem badań jest sprawdzenie jak zachowa się model przy większych rozmiarach słownika. Przy badańach wstępnych 
wynosiła ona \begin{math}20000\end{math}. W tym eksperymencie użyje rozmiaru słownika takiego samego jak użyli w 
swojej pracy autorzy \cite{hellendoorn} wynoszącego 74000. Dla takiego rozmiaru pomijane są słowa 
występujące rzadziej niż 5 razy. wszystkie pozostałe parametry pozostają takie same.
\begin{table}[ht]
    \centering
    \begin{tabular}{|l|l|}
        \hline
        \multicolumn{1}{|c|}{Miara}  & Wynik\\ \hline
        Top 1                            & 0.30 \\ \hline
        Top 3                            & 0.52 \\ \hline
        Top 5                            & 0.61 \\ \hline
        Top 5 z kolejnością              & 0.42 \\ \hline
        Top 10 z kolejnością             & 0.43 \\ \hline
        najmniejsza wartość funkcji celu & 1.75 \\ \hline
        \end{tabular}
    \caption{Wyniki uzyskane przez model o dużym słowniku} 
    \label{wyniki_duzego}

\end{table} \\ 

\subsection{Zastosowanie warstwy gru}
Ostatnim wykonanym przeze mnie eksperymentem jest porównanie skuteczności warstwy gru z warstwą lstm. Badanie to przeprowadzam 
dla parametrów wyróżnionego modelu \ref{wyrozniony} podmieniając rodzaj warstwy rnn. Uzyskane wyniki zostały przedstawione w tabeli \ref{wyniki_gru}. 
\begin{table}[ht]
    \centering
    \begin{tabular}{|l|l|}
        \hline
        \multicolumn{1}{|c|}{Miara}  & Wynik\\ \hline
        Top 1                            & 0.32 \\ \hline
        Top 3                            & 0.59 \\ \hline
        Top 5                            & 0.68 \\ \hline
        Top 5 z kolejnością              & 0.46 \\ \hline
        Top 10 z kolejnością             & 0.47 \\ \hline
        najmniejsza wartość funkcji celu & 1.57 \\ \hline
        \end{tabular}
    \caption{Wyniki uzyskane przez model o warstwie GRU} 

    \label{wyniki_gru}
\end{table} \\ 

\subsection{przegląd hiperparametrów}
\begin{description}
    \item[Wpływ długości sekwencji wejściowe] 
    \hfill \\ Model został wytrenowany na czterech długościach sekwencji wejściowej: \begin{math}1, 5, 10, 15\end{math}. Można
    zaobserwować wzrost skuteczności modelu wraz ze wzrostem sekwencji, do 10 słów. dalsze próby powiększania sekwencji wpływają negatywnie 
    na jego działanie. spowodowane jest to tym, że dla krótkich sekwencji takich jak \begin{math}1, 5\end{math} model posiada za mało 
    kontekstu aby wykonać predykcję, natomiast sekwencje długości 15 są już zbyt specyficznymi częściami konkretnego programu aby móc je 
    uogólnić dla przewidywanego kodu. \\
    \item[Wpływ długości rozmiaru słownika] 
    \hfill \\ Wielkość słownika ma duży wpływ na działanie modelu. Przy małych wielkościach można spodziewać się niepoprawnego działania 
    ze względu na to, że sieć nie będzie znała słów, które ma przewidywać. 
    jak pokazują wyniki eksperymentu w tabeli \ref{wyniki_duzego}, słownik o bardzo dużej liczbie słów również pogarsza działanie algorytmu. 
    Różnica ta wynosi około \begin{math}5 \end{math} punktów procentowych we wszystkich kategoriach. Duże rozmiary słownika również znaczenie spowalniają 
    proces treningu. \\
    \item[Liczba warstw sieci rekurencyjnej]
    \hfill \\ Eksperymenty zostały przeprowadzone dla jednej oraz dwóch warstw sieci rekurencyjnej. 
    We wszystkich przeprowadzonych eksperymentach, modele o dwóch warstwach osiągnęły podobne lub lepsze wyniki od modeli o jednej warstwie. \\
    \item[Wpływ liczby neruronów w warstwie rekurencyjnej]
    \hfill \\ Eksperymenty zostały przeprowadzone dla warstw składających cie z \begin{math} \end{math} warstw sieci rekurencyjnej. Można 
    zaobserwować, że duże liczby neuronów w warstwie nie poprawiają działania modelu. Najlepszy rezultat oraz inne rezultaty zbliżone 
    do niego uzyskują warstwy składające się ze 128 neuronów. \\
    \item[Wpływ warstwy CNN]
    \hfill \\ Przeprowadziłem dwa eksperymenty z zastosowaniem warstwy CNN, oba dla jednej warstwy LSTM o 128 neuronach różniących się długością 
    sekwencji wejściowe. Po porównaniu tych modeli z modelami o tych samych parametrach bez warstwy CNN, zauważalna jest nieznaczna poprawa, około 
    \begin{math}1\end{math} punkt procentowy dla modelu o długości sekwencji równej \begin{math}5\end{math}, oraz około \begin{math}10\end{math} 
    punktów procentowych dla modelu o długości wejściowej równej 10. Mimo poprawy modele te wciąż osiągają znacznie gorsze wyniki od 
    pozostałych badanych modeli. \\
    \item[Zastosowanie warstwy GRU]
    \hfill \\ Zastosowanie warstwy GRU poprawiło działanie wyróżnionego modelu w kategoriach Top 3 oraz Top 5, natomiast pogorszyło w Top 1. Różnice te są jednak bardzo 
    niewielkie, wynoszą około \begin{math}2 \end{math} punktów procentowych i najprawdopodobniej wynikają z losowości procesu uczenia. Można 
    wywnioskować zatem, że w tak przedstawionym zadaniu rodzaj zastosowanej warstwy nie ma wpływu na skuteczność modelu. Warto zatem użyć warstwy 
    GRU w celu skrócenia czasu treningu. \\
  \end{description}

\subsection{Najlepsza znaleziona architektura}
  Jako, że model trenowany jest z myślą o stworzeniu wtyczki ułatwiającej pisanie kodu, do wybrania najlepszej architektury posłużę się 
  metryką uwzględniającą 5 najlepszych podpowiedzi z uwzględnieniem kolejności wystąpienia. W tym kryterium oba modele scharakteryzowane 
  w sekcjach \ref{wyniki_gru}, \ref{wyniki_najlepszego} uzyskują taką samą skuteczność wynoszącą \begin{math}46\%\end{math}, jednak dla 
  5 najlepszych parametrów bez uwzględnia kolejności radzi sobie lepiej model przedstawiony w \ref{wyniki_gru} osiągając skuteczność 
  równą \begin{math}68\%\end{math}, zatem to on zostanie 
  użyty w implementacji wtyczki oraz porównania z architekturami z innych prac. W tabeli \ref{najelpszy} przedstawiam dokładne parametry tego modelu. 
  \begin{table}[ht]
    \centering
    \begin{tabular}{|l|l|}
        \hline
        \multicolumn{1}{|c|}{parametr} & wartość           \\ \hline
        liczba warstw rnn              & 2                 \\ \hline
        rodzaj rnn                     & GRU                \\ \hline
        liczba neuronów w warstwie     & 128               \\ \hline
        długość sekwencji wejściowej   & 10                \\ \hline
        liczba trenowanych parametrów  & 3,434,177         \\ \hline
        rozmiar słownika               & 20000             \\ \hline
        rozmiar wektora zanurzeń       & 32                \\ \hline
        rozmiar porcji danych          & 128               \\ \hline
        liczba epok                    & 25                \\ \hline
        optymalizator                  & adam              \\ \hline
        funkcja celu                   & entropia krzyżowa \\ \hline
        \end{tabular}
    \caption{Wybrany najlepszy model} 

    \label{najelpszy}
\end{table} 

Na wykresie \ref{wykres} została przedstawiona wartość funkcji celu w kolejnych epokach treningu. 
\begin{figure}[!h]
	% Znacznik \caption oprócz podpisu służy również do wygenerowania numeru obrazka;
    \caption{Wartość funkcji celu w kolejnych epokach}

	% dlatego zawsze pamiętaj używać najpierw \caption, a potem \label
    \label{wykres}
    % Zamiast width można też użyć height, etc. 
    \centering \includegraphics[width=103mm, height=76mm]{wykres.png}
\end{figure}

\subsection{Generowane wyniki}
\begin{figure}[!h]
	% Znacznik \caption oprócz podpisu służy również do wygenerowania numeru obrazka;
    \caption{Przykładowy wynik wygenerowany przez model}

	% dlatego zawsze pamiętaj używać najpierw \caption, a potem \label
    \label{fig:wynik}
    % Zamiast width można też użyć height, etc. 
    \centering \includegraphics[width=103mm, height=76mm]{wynik.png}
\end{figure}
Na rysunku \ref{fig:wynik} został przedstawiony przykładowy program którego uzupełnienie realizował model. Na zielono zostały zaznaczone tokeny,
które znalazły się w 5 pierwszych predykcjach modelu, pozostałe tokeny oznaczone zostały kolorem czerwonym. Można zaobserwować, że 
model w bardzo dobrym stopniu radzi sobie z przewidywaniem struktury programu, uzupełnianiem nawiasów, wykrywaniem końca linii, stawianiem słów kluczowych. 
Gorzej radzi sobie z uzupełnianiem nazw pochodzących z bibliotek oraz definiowanych zmiennych. 

\subsection{Porównanie uzyskanych wyników}
Modele pochodzące z przytoczonych prac pełnią to samo zadanie, jednak trenowane są na innych zbiorach danych. Mimo to, 
uzyskany przeze mnie model osiąga bardzo zbliżone do nich wyniki, oraz warto zestawić je ze sobą. 

Autor \cite{erik} uzyskał model o skuteczności wynoszącej \begin{math}69.7\%\end{math}, dla 
5 najlepszych wyników. Jest to niemal identyczny wynik z modelem uzyskanym przeze mnie. Mimo to architektury 
modeli znacznie różnią się, gdyż w pracy zastosował jedną warstwę LSTM o 512 neuronach. Różnica ta może wynikać z narzucenia innych 
hiperparametrów. 

Autor \cite{contextual_code_completion} uzyskał model o skuteczności wynoszącej \begin{math}66.3\%\end{math} dla 3 najlepszych predykcji, bez uwzględniania 
kolejności ich wystąpienia. Skuteczność mojego modelu w tej kategorii jest mniejsza, gdyż wynosi ona \begin{math}0.59\%\end{math}. Różnica 
ta najprawdopodobniej wynika z tego, że model przedstawiony w pracy \cite{contextual_code_completion} jest trenowany oraz testowany na pojedyńczych bibliotekach, 
zatem realizuje on znacznie prostsze zadanie. Modele te również wykorzystują całkowicie inną architekturę. Wykorzystany zostaje model ze skupieniem uwagi składający 
się z 3 nieliniowych warstw. 

W pracy autorzy \cite{hellendoorn} uzyskują model o skuteczności wynoszącej \begin{math}67.9\%\end{math}, dla 10 najlepszych predykcji z uwzględnieniem 
kolejności ich wystąpienia. Mój model osiąga w tej metryce gorszy wynik, równy \begin{math}47\%\end{math}. Różnice te mogą wynikać z znacznie 
dłuższego czasu treningu wynoszącego około 3 dni, oraz z bardziej złożonego mechanizmu predykcji, uwzględniającego zagnieżdżenia w kodzie. 

\subsection{Możliwe błędy}
Wszystkie oceniane przeze mnie modele trenowane były tylko raz. Ze względu na długi czas wykonania pojedyńczego eksperymentu, zdecydowałem 
się na porównanie ze sobą wiele różnych modeli zamiast kilku modeli kilkakrotnie trenowanych. Efektem tego mogą być nieprecyzyjne wyniki 
wynikające z losowości procesu uczenia, gdyż rozmiar próbki równy 1, jest zdecydowanie za mały aby dokładnie określić działanie modelu. 
Jako, że nie posiadamy jeszcze wystarczającej wiedzy na temat cech wspólnych pomiędzy językami programowania a językami naturalnymi, 
sugerowanie się modelami pochodzącymi z tej dziedziny nie przynosi odpowiednich rezultatów. 

Wszystkie zastosowane miary traktują przewidziane tokeny równo. Nie oddają one zatem dokładnie w jakim stopniu wtyczka usprawnia pracę 
programisty, gdyż przewidzenie długiej nazwy oszczędza znacznie więcej czasu niż np. pojedyńczego przecinka. 

Relacja pomiędzy kryteriami, którymi są oceniane modele, a faktyczną skutecznością tych modeli w praktyce jest nie znana. Zbyt duża liczba 
sugestii proponowanych przez model również negatywnie wpływa na pracę programisty, ponieważ sprawia, że pole zawierające sugestie staje 
się nieczytelne. Ocena z uwzględnieniem kolejności zakłada, że predykcja znajdująca się na drugim miejscu jest tylko w połowie tak użyteczna, 
jak ta znajdująca się na miejscu pierwszym, natomiast sugestia na trzecim miejscu tylko w jednej trzeciej itd. Założenie to wynika z tego, że 
wybranie niższej oceny wymaga więcej pracy od programisty. Mimo to łatwo pominąć ten problem np. stosując skróty klawiszowe, zatem 
nie jesteśmy w stanie określić, czy to założenie jest poprawne. 






\newpage % Rozdziały zaczynamy od nowej strony.
 

\section{Implementacja wtyczki}
\begin{description}
\item[Zintegrowane środowisko programistyczne Sublime Text 3]
\hfill\\
\begin{figure}[!h]
	% Znacznik \caption oprócz podpisu służy również do wygenerowania numeru obrazka;
	\caption{Sublime Text 3}
	% dlatego zawsze pamiętaj używać najpierw \caption, a potem \label
    \label{fig:sublime_logo}
    % Zamiast width można też użyć height, etc. 
    \centering \includegraphics[width=54mm, height=54mm]{sublime_logo.png}
\end{figure}
Sublime Text 3 \cite{sublime} jest wieloplatformowym, rozbudowanym i wysoce konfigurowalnym edytorem teksty stworzonym z myślą 
o programistach. Udostępnia ono interfejs programistyczny w języku Python pozwalający na proste tworzenie własnych 
wtyczek lub instalacje wtyczek stworzonych przez społeczność użytkowników aplikacji, jak i modyfikowanie samego 
środowiska. Natywnie wspiera obsługę wszystkich najpopularniejszych języków programowania oraz języków znaczników 
(markup language) co czyni je bardzo uniwersalnym oraz zawdzięcza temu swoją dużą popularność. Na liście najczęściej 
wybieranych środowisk programistycznych \cite{topide} ocenianych pod kątem częstości odwiedziń strony pobierania, 
aktualnie zajmuje 9 miejsce. Wyprzedają je głównie środowiska tworzone pod kątem rozwoju w konkretnym języku programowania 
takie jak pyCharm lub Eclipse.  

Sublime Text jest zaimplementowane w języku Python oraz C++. \\

\item[Architektura wtyczki]
\hfill\\ 
W celu uniknięcia konieczności doinstalowywania przez użytkowników dodatkowych modułów do wirtualnego środowiska wykonawczego programu Sublime Text, 
zaimplementowałem wtyczkę w architekturze klient-serwer. Wtyczka przy uruchomieniu programu Sublime Text uruchamia zewnętrzny proces serwera 
przyjmującego zapytania pod lokalnym adresem komputera na porcie 8000. Serwer składa się najlepszego znalezionego przeze mnie modelu 
uczenia głębokiego oraz kombinacji unigramu i bigramu. Podejście to pozwala również na uniknięcie kombinacji związanych z niezgodnościami 
wersji kompilatora Python środowiska Sublime Text (Python 3.6) z wersjami użytych przeze mnie bibliotek 

\begin{figure}[!h]
	% Znacznik \caption oprócz podpisu służy również do wygenerowania numeru obrazka;
	\caption{Architektura wtyczki}
	% dlatego zawsze pamiętaj używać najpierw \caption, a potem \label
    \label{fig:klient-serwer}
    % Zamiast width można też użyć height, etc. 
    \centering \includegraphics[width=450px, height=200px]{klient_serwer.png}
\end{figure}
\end{description}
\subsection{Obsługa}
Wtyczka po każdym naciśnięciu spacji, enter lub któregoś ze znaku specjalnego wysyła zapytanie do serwera składające sie z całego kodu aktualnie 
modyfikowanego programu, a w odpowiedzi uzyskuje wiadomość składającą sie z 5 najlepszych predykcji wykonanych przez model. Następnie użytkownik 
może wybrać którąś z sugestii poprzez naciśnięcie kombinacji klawiszy \begin{math}ctrl + i, i\in \{1,2,3,4,5\}\end{math}.
\begin{figure}[!h]
	% Znacznik \caption oprócz podpisu służy również do wygenerowania numeru obrazka;
	\caption{Działanie wtyczki}
	% dlatego zawsze pamiętaj używać najpierw \caption, a potem \label
    \label{fig:dzialanie_wtyczki}
    % Zamiast width można też użyć height, etc. 
    \centering \includegraphics[width=450px, height=300px]{ss.png}
\end{figure}
\subsection{Czas odpowiedzi modelu}
Po wykonaniu 1000 zapytań do serwera odpowiedzialnego za wykonywanie predykcji, średni czas odpowiedzi wyniósł 18.8ms. W skład tej operacji 
wchodzi wysłanie zapytania do serwera lokalnego, tokenizacja danych, wykonanie predykcji oraz odesłanie odpowiedzi. Czas ten jest na tyle 
krótki, że nie ma żdanego wpływu na efektywność pisania kodu. 



\newpage % Rozdziały zaczynamy od nowej strony.
 

\section{Podsumowanie}


  
  \begin{description}
    \item[Omówienie] 
    \hfill \\ W swojej pracy przedstawiam zastosowanie rekurencyjnych sieci neuronowych do autouzupełniania kodu źródłowego. W swoich 
    eksperymentach uzyskałem model osiągający skuteczność równą \begin{math}68\%\end{math} dla 5 pierwszych predykcji oraz 
    \begin{math}46\%\end{math} po uwzględnieniu kolejności ich wystąpienia. Badam również wpływ hiperparametrów takich jak 
    długość sekwencji wejściowej, rodzaje warstw sieci rekurencyjnych, układ warstw sieci, rozmiar słownika i
    ilość neuronów w warstwie. Pokazuję, że w zadaniu przewidywania kodu lepiej działają modele, w których skład wchodzi 
    kilka warstw o małej ilości neuronów oraz długości sekwencji wejściowych liczące 10 słów. \\

    \item[Napotkane problemy]
    \hfill\\
        \label{chellenges}
    Głównym wyzwaniem oraz detalem różniącym języki programowania od języków naturalnych, jest możliwość nadawania dowolnych
    nazw obiektom oraz metodom, przez co nie można objąć wszystkich słów w słowniku danych treningowych. Słowa tego typu 
    nazywane są słowami poza słownikiem. Zwiększanie wielkości słownika nigdy nie obejmie wszystkich możliwych nazw, natomiast 
    bardzo spowolni ostatni krok algorytmu, którym jest obliczenie wyznaczenie funkcji softmax. Jak zostało pokazane w publikacji 
    \cite{hellendoorn} od pewnego momentu większy rozmiar słownika zaczyna wpływać negatywnie na skuteczność modelu. \\


    Nadmierne dopasowanie modelu do danych treningowych może wystąpić przy zbyt długim treningu. Taki model 
    zacznie dawać bardzo dobre predykcje na zbiorze treningowym jednak bardzo słabo poradzi sobie na zbiorze 
    walidacyjnym. Zamiast zgeneralizować problem model nauczy się danych treningowych 'na pamięć'.\\


    Ostatnim napotkanym przeze mnie problemem były ograniczenia wynikające ze sprzętu. Aby trening sieci skończył się w 
    sensownym czasie musi się on odbywać na karcie graficznej. Nie ma serwisu udostępniającego moc obliczeniową tych 
    kart za darmo na wystarczająco długi czas.  \\
    \item[Dalsze prace] 
    \hfill \\ 
\begin{description}
    \item[Poprawa modelu] 
    \hfill \\ Dalsze badania dotyczące hiperparametrów np. długości sekwencji wejściowych z przedziału (5, 10) i (10, 15) jak i 
    wielokrotne treningi modelu mogą znacznie wpłynąć na poprawę uzyskanych w tej pracy wyników. Ze względu na ograniczenia czasowe, 
    modele testowane w tej pracy były trenowane przez względnie krótki czas. Wydłużenie czasu treningu również powinno pozytywnie 
    wpłynąć na model końcowy. 
     \\
    \item[Inne języki programowania] 
    \hfill \\ Ze względu na bardzo duże podobieństwa między językami programowania, warto zbadać zachowanie przedstawionych modeli 
    na językach innych niż Python. Zastosowane w tej pracy rozwiązania bardzo łatwo adaptują się do eksperymentów z innymi danymi
    wejściowymi. Możemy oczekiwać innych rezultatów dla języków statycznych takich jak Java lub języków deklaratywnych np. SQL. \\
    \item[Rozszerzenie dla innych środowisk programowania]
    \hfill \\ Wtyczka w tej pracy została zaimplementowana tylko dla środowiska SublimeText 3, jednak ze względu na jej prostą oraz 
    rozszerzalną budowę bardzo łatwo wprowadzić ją do innych środowisk programistycznych. Dzięki architekturze serwerowej może obsługiwać 
    wiele klientów jednocześnie, oraz po przeniesieniu do chmury nie wymagałaby pobierania sieci neuronowej i bibliotek uczenia maszynowego 
    na komputer użytkownika.\\
  \end{description}

  \end{description}


%--------------------------------------------
% Literatura
%--------------------------------------------
\cleardoublepage % Zaczynamy od nieparzystej strony
\printbibliography

%--------------------------------------------
% Spisy (opcjonalne)
%--------------------------------------------
\newpage
\pagestyle{plain}

% Wykaz symboli i skrótów.
% Pamiętaj, żeby posortować symbole alfabetycznie
% we własnym zakresie. Ponieważ mało kto używa takiego wykazu,
% uznałem, że robienie automatycznie sortowanej listy
% na poziomie LaTeXa to za duży overkill.
% Makro \acronymlist generuje właściwy tytuł sekcji,
% w zależności od języka.
% Makro \acronym dodaje skrót/symbol do listy,
% zapewniając podstawowe formatowanie.
% //AB
\vspace{0.8cm}
\acronymlist
\acronym{EiTI}{Wydział Elektroniki i Technik Informacyjnych}
\acronym{PW}{Politechnika Warszawska}

\listoffigurestoc     % Spis rysunków.
\vspace{1cm}          % vertical space
\listoftablestoc      % Spis tabel.
\vspace{1cm}          % vertical space
\listofappendicestoc  % Spis załączników

% Załączniki
% \newpage
% \appendix{Nazwa załącznika 1}
% \lipsum[1-8]

% \newpage
% \appendix{Nazwa załącznika 2}
% \lipsum[1-4]

% Używając powyższych spisów jako szablonu,
% możesz tu dodać swój własny wykaz bądź listę,
% np. spis algorytmów.

\end{document} % Dobranoc.
